% -*- coding: utf-8; -*-

\chapter{Trabalhos Relacionados}
\label{ch:related}
	Desde a publicação de \quote{Semi-automatic Generation of Transfer Functions for Direct Volume Rendering} em 1998, \textit{Kindlmann e Durkin}~\cite{gordon} vêm sendo citados quando o assunto é função de transferência. Seu trabalho (explicado em maiores detalhes no capítulo~\ref{ch:gordon}) utiliza a mesma ideia dos detectores de aresta para encontrar fronteiras, onde uma aresta se encontra nos pontos de primeira derivada máxima e segunda igual a zero, como em \textit{Canny}~\cite{canny}. Com o modelo matemático de uma fronteira ideal e suas derivadas, uma relação entre um valor escalar e sua distância até o centro da fronteira é extraída. Para compor a função de transferência, o usuário fornece uma função que relaciona opacidade e distância à fronteira. 
	
	Um histograma 2D acumula a ocorrência de todos os pares de valores do volume e magnitude do gradiente, que são respectivamente os eixos \textit{x} e \textit{y}. O histograma ajuda a identificar as fronteiras. Elas aparecem como arcos que vão de um valor escalar ao outro. Em volumes mais complexos é muito comum arcos se sobreporem. Isso indica uma falha na classificação destes valores, pois não é possível definir a qual fronteira eles realmente pertencem. A fim de minimizar essa sobreposição, uma versão 2D da função de transferência também é proposta, utilizando a primeira derivada de cada voxel no lugar de sua média.
	
	Em 2001, \textit{Kniss et al.}~\cite{kniss1} estenderam essa metodologia para funções de transferência 3D, também buscando classificar melhor as fronteiras. A complexidade da FT 3D gerou a necessidade de aprimorar a interface com o usuário, fornecendo ferramentas de inspeção de dados no volume com resposta direta na função de transferência. Dessa forma era possível saber que região da FT correspondia à porção do volume selecionada. No ano seguinte \textit{Kniss et al.}~\cite{kniss2} aprimoraram ainda mais o controle do usuário sobre a função de transferência.
	
	Anos depois \textit{Park e Bajaj}~\cite{park} utilizaram traçado de raios para eliminar as médias de \textit{Kindlmann e Durkin}~\cite{gordon} e obter uma função de transferência espacial, onde a opacidade é calculada individualmente para cada voxel. No mesmo ano, também buscando resolver a sobreposição de regiões, \textit{Lum e Ma}~\cite{lumema} optaram por mudar o domínio da função de transferência. Neste método, para cada voxel do volume registra-se em um histograma 2D o par de vizinhos mais próximos na direção do gradiente. Os dois eixos do histograma são horizontais, sendo um inferior no sentido crescente dos valores e um superior decrescente. Dessa forma, as fronteiras são identificadas como segmentos retos que ligam os eixos e não mais arcos como em~\cite{gordon}.
	
	Muitos trabalhos seguiram esse mesmo caminho, em busca de um novo domínio para a função de transferência. É o caso de \textit{Sereda et al.}~\cite{sereda1}, que identificam os valores extremos de uma fronteira (FL e FH) através de 3 possíveis perfis de função. As ocorrências de FL e FH são acumuladas em seu \quote{Histograma LH}: um histograma 2D aonde as fronteiras ideais se apresentam como regiões circulares acima da reta FL = FH. Um algoritmo de crescimento de região é proposto para agrupar todas as entradas do histograma que pertencem à mesma fronteira. Caso o usuário queira ver mais de uma região, ele deve indicar sementes para o algoritmo.
	
	Outro exemplo de um novo domínio para a função de transferência é apresentado por \textit{Haidacher et al.}~\cite{haidacher}. Eles propõem acumular em um histograma 2D a média e o desvio padrão de todos os voxels. Mas assim como em \textit{Sereda et al.}~\cite{sereda1} o histograma apresenta as fronteiras e os materiais entre elas de forma visualmente semelhante, forçando o usuário ao processo de tentativa e erro.
	
	Gradativamente os trabalhos seguintes desta área foram diminuindo o foco em automatizar a geração de funções de transferência, como visto em~\cite{zou, wang} que propõem novas ferramentas de seleção para o histograma 2D proposto por \textit{Kindlmann e Durkin}~\cite{gordon}.
	
	Contudo, nem todas as pesquisas partiram do princípio da detecção de arestas. Na mesma época em que \textit{Kniss et al.} apostaram em interface e ferramentas para o usuário, \textit{Pekar et al.}~\cite{pekar} propuseram outras curvas médias para identificar as fronteiras de um volume, como área da isosuperfície, curvatura da isosuperfície e gradiente total dividido pela curvatura. De forma semelhante, \textit{Tenginakai et al.}~\cite{salient} também não abordam a geração de Funções de transferência em si, mas propõem revelar quais valores do volume correspondem a fronteiras. Neste, Higher Order Moments (momentos estatísticos de alta ordem) são calculados localmente para cada voxel e depois acumulados em histogramas. A segunda e terceira ordem respondem à fronteira de forma semelhante à primeira e segunda derivadas em \textit{Kindlmann e Durkin}~\cite{gordon}, mas outros padrões também são identificados em outras ordens. A \quote{skew} identifica uma fronteira ao cruzar o zero enquanto na \quote{kurtosis} o comportamento esperado é um vale no histograma.
	
	Alguns trabalhos buscaram uma melhor classificação do volume através de coerência espacial. Em 2006, \textit{Lundstrom et al.}~\cite{lundstrom1} introduziram o conceito de \textit{Partial Range Histograms} (PRH), um histograma parcial que acumula apenas valores a um certo alcance de um valor referencial. Para compor o PRH os valores também tem que vir de voxels próximos àqueles com o valor de referência. A ideia central é reduzir o histograma comum do volume e criar um outro histograma que acumula PRHs. Uma gaussiana é estimada para reproduzir cada pico do histograma comum. A partir da largura da gaussiana, o conjunto de valores para o PRH daquele pico é definido. Então, a gaussiana é subtraída do histograma global e o resultado do PRH é acumulado. A solução apresenta bons resultados em resolver a sobreposição de regiões. Porém, o acumulador de PRHs não apresenta a localização da fronteira com clareza.
	
	Outros métodos atingiram visualizações muito boas e apresentaram abordagens bastante automáticas, mas ao mesmo tempo que livraram o usuário do processo de tentativa e erro também dificultaram sua intervenção sobre a função de transferência gerada, como em~\cite{ruiz, zhou}.
	
	Recentemente, o trabalho de \textit{Lan et al.}~\cite{lan} retomou o domínio da função de transferência utilizado por \textit{Kindlmann e Durkin}~\cite{gordon}. O problema de sobreposição de regiões é resolvido através de conectividade espacial e crescimento de região (quando necessário). Sua proposta não é identificar fronteiras entre diferentes materiais, ou gerar funções de transferência para elas. O objetivo do trabalho é separar todas as estruturas espaciais do volume e permitir ao usuário determinar sua cor e opacidade através da interface. 
	
	Apesar de apresentar uma opção bastante completa ao usuário, o trabalho de \textit{Lan et al.} não melhora a detecção automática da função de transferência. Ele apenas separa os isovolumes que resultam de~\cite{gordon, kniss1, kniss2}. Dessa forma, este trabalho aborda um problema ainda em aberto: gerar automaticamente funções de transferência que realcem todas fronteiras de um volume de dados, permitindo ao usuário um controle fino da FT gerada, através de uma interface simples.