% -*- coding: utf-8; -*-

\chapter{Trabalhos Relacionados}
\label{ch:related}
	Desde a publicação de \quote{Semi-automatic Generation of Transfer Functions for Direct Volume Rendering} em 1998, \textit{Kindlmann e Durkin}~\cite{gordon} vêm sendo citados quando o assunto é função de transferência. Seu trabalho (explicado em maiores detalhes no Capítulo~\ref{ch:gordon}) utiliza a mesma ideia dos detectores de aresta para encontrar fronteiras, onde uma aresta se encontra nos pontos de primeira derivada máxima e segunda derivada igual a zero, como em \textit{Canny}~\cite{canny}. Com o modelo matemático de uma fronteira ideal e suas derivadas, uma relação entre um valor escalar e sua distância até o centro da fronteira é extraída. Para compor a função de transferência, o usuário fornece uma função que relaciona opacidade e distância à fronteira. 
	
	Um histograma 2D que acumula a ocorrência de todos os valores de magnitude do gradiente em função do valor de intensidade do volume, ajuda a identificar as fronteiras existentes. Elas aparecem como arcos que vão de um valor de intensidade ao outro. Em volumes mais complexos é muito comum arcos se sobreporem. Isso indica uma falha na classificação destes valores, pois não é possível definir a qual fronteira eles realmente pertencem. A fim de minimizar essa sobreposição, uma versão 2D da função de transferência também é proposta, onde a opacidade não varia apenas com a intensidade do volume, mas também com a magnitude do seu gradiente.
	
	Em 2001, \textit{Kniss et al.}~\cite{kniss1} estenderam essa metodologia para funções de transferência 3D, também buscando classificar melhor as fronteiras. A complexidade da FT 3D gerou a necessidade de aprimorar a interface com o usuário, fornecendo ferramentas de inspeção de dados no volume com resposta direta na função de transferência. Dessa forma era possível saber que região da FT correspondia à porção do volume selecionada. No ano seguinte \textit{Kniss et al.}~\cite{kniss2} aprimoraram ainda mais o controle do usuário sobre a função de transferência.
	
	Anos depois, \textit{Park e Bajaj}~\cite{park} utilizaram traçado de raios para eliminar as médias de \textit{Kindlmann e Durkin}~\cite{gordon} e obter uma função de transferência espacial, onde a opacidade é calculada individualmente para cada voxel. Apesar dessa abordagem resultar em uma visualização mais fiel ao volume, ela torna os ruídos mais perceptíveis, podendo apresentar artefatos que não são reais.
	
	No mesmo ano, também buscando resolver a sobreposição de regiões, \textit{Lum e Ma}~\cite{lumema} optaram por mudar o domínio da função de transferência. Neste método, para cada voxel do volume registra-se em um histograma 2D o par de vizinhos mais próximos na direção do gradiente. Os dois eixos do histograma são horizontais, sendo um inferior no sentido crescente dos valores e um superior decrescente. Dessa forma, as fronteiras são identificadas como segmentos retos que ligam os eixos e não mais arcos como em~\cite{gordon}.
	
	Muitos trabalhos seguiram esse mesmo caminho, em busca de um novo domínio para a função de transferência. É o caso de \textit{Sereda et al.}~\cite{sereda1}, que identificam uma fronteira através de 3 possíveis perfis de função, que descrevem a variação da intensidade do volume de um valor escalar menor (FL) a um valor escalar maior (FH). As ocorrências de FL e FH são acumuladas em seu \quote{Histograma LH}: um histograma 2D aonde as fronteiras ideais se apresentam como regiões circulares acima da reta FL = FH. Um algoritmo de crescimento de região é proposto para agrupar todas as entradas do histograma que pertencem à mesma fronteira. Caso o usuário queira ver mais de uma região, ele deve indicar sementes para o algoritmo.
	
	Outro exemplo de um novo domínio para a função de transferência é apresentado por \textit{Haidacher et al.}~\cite{haidacher}. Eles propõem acumular em um histograma 2D a média e o desvio padrão de todos os voxels. Mas assim como em \textit{Sereda et al.}~\cite{sereda1}, o histograma apresenta as fronteiras e os materiais entre elas de forma visualmente semelhante, forçando o usuário ao processo de tentativa e erro para selecionar apenas as fronteiras.
	
	Gradativamente os trabalhos seguintes dessa área foram diminuindo o foco em automatizar a geração de funções de transferência, como visto em~\cite{zou, wang} que propõem novas ferramentas de seleção para o histograma 2D proposto por \textit{Kindlmann e Durkin}~\cite{gordon}.
	
	Contudo, nem todas as pesquisas partiram do princípio da detecção de arestas. Na mesma época em que \textit{Kniss et al.} apostaram em interface e ferramentas para o usuário, \textit{Pekar et al.}~\cite{pekar} propuseram outras curvas médias para identificar as fronteiras de um volume, como área da isosuperfície, curvatura da isosuperfície e gradiente total dividido pela curvatura. De forma semelhante, \textit{Tenginakai et al.}~\cite{salient} também não abordam a geração de funções de transferência em si, mas propõem revelar quais valores do volume correspondem a fronteiras. Neste, Higher Order Moments (momentos estatísticos de alta ordem) são calculados localmente para cada voxel e depois acumulados em histogramas. A segunda e terceira ordem respondem à fronteira de forma semelhante à primeira e segunda derivadas em~\cite{gordon}, mas outros padrões também são identificados em outras ordens. A \quote{skew} identifica uma fronteira ao cruzar o zero enquanto na \quote{kurtosis} o comportamento esperado é um vale.
	
	Alguns trabalhos buscaram uma melhor classificação do volume através de coerência espacial. Em 2006, \textit{Lundstrom et al.}~\cite{lundstrom1} introduziram o conceito de \textit{Partial Range Histograms} (PRH), um histograma parcial que acumula apenas valores a um certo alcance de um valor referencial e dentro de uma vizinhança conhecida. A ideia central é reduzir o histograma total do volume, pico a pico e construir um novo histograma que acumula PRHs. Uma gaussiana é estimada para reproduzir cada pico do histograma total. A partir da largura da gaussiana, o conjunto de valores para o PRH daquele pico é definido. Então, a gaussiana é subtraída do histograma total e o resultado do PRH é acumulado no novo histograma. A solução apresenta uma boa classificação dos voxels do volume quanto ao material, o que facilita a especificação manual da função de transferência para revelar as diferentes regiões do volume, mas não as interfaces entre elas.
	
	Outros métodos também apresentaram abordagens automáticas, porém, menos flexíveis. Ao mesmo tempo que livraram o usuário do processo de tentativa e erro, também dificultaram sua intervenção sobre a função de transferência gerada, como em~\cite{ruiz, zhou}.
	
	Recentemente, o trabalho de \textit{Lan et al.}~\cite{lan} retomou o domínio da função de transferência utilizado por \textit{Kindlmann e Durkin}~\cite{gordon}. O problema de sobreposição de regiões é resolvido através de conectividade espacial e crescimento de região (quando necessário). Sua proposta não é identificar fronteiras entre diferentes materiais, ou gerar funções de transferência para elas. O objetivo do trabalho é separar todas as estruturas espaciais do volume e permitir ao usuário determinar sua cor e opacidade através da interface.
	
	Ainda em 2016, um trabalho com objetivo semelhante foi proposto por \textit{Ponciano et al.}~\cite{marroquim}. Nele, o volume é automaticamente classificado em diferentes regiões através de seu histograma 2D. Inicialmente, um grafo é gerado com todas as células do histograma, onde o peso das arestas é definido pela persistência, altura, distância e área das células. Em seguida, uma árvore geradora mínima é criada, unindo as áreas do histograma que representam as regiões do volume. Então, uma interface permite que o usuário delete, separe e una áreas do histograma, bem como alterar sua cor e opacidade.
	
	%TODO CONCLUIR FALANDO DOS DOIS ÚLTIMOS
	Apesar de apresentar uma opção bastante completa ao usuário, o trabalho de \textit{Lan et al.} não melhora a detecção automática da função de transferência. Ele apenas separa os isovolumes que resultam de~\cite{gordon, kniss1, kniss2}. Dessa forma, este trabalho aborda um problema ainda em aberto: gerar automaticamente funções de transferência que realcem todas fronteiras de um volume de dados, permitindo ao usuário um controle fino da FT gerada, através de uma interface simples.