% -*- coding: utf-8; -*-

\chapter{Conclusão e Trabalhos Futuros}
\label{ch:conclusion}
	Esta dissertação propôs um novo método semiautomático para obter funções de transferência, baseado na média da terceira derivada do volume, em função de seus valores escalares. Este método, inspirado no trabalho de \textit{Kindlmann e Durkin}~\cite{gordon} (com o qual é comparado), mostrou resultados equivalentes e superiores quanto à exatidão do centro das fronteiras detectadas. Em alguns volumes, o método destaca fronteiras em excesso, paralelas entre si, mas ao mesmo tempo, encontra fronteiras importantes não detectadas pelo método de \textit{Kindlmann e Durkin}.

	Foi demonstrado que o trabalho de \textit{Kindlmann e Durkin}~\cite{gordon} pode gerar funções de transferência deslocadas, onde as isosuperfícies destacadas não estão alinhadas com os centros das fronteiras reais. Destacou-se também a importância de se encontrar um valor apropriado de $ g_{thresh} $ para poder obter uma visualização apropriada, mostrando que a necessidade desse parâmetro diminui o caráter automático do trabalho.
	
	O método que aqui foi proposto não incorporou $ g_{thresh} $ nem a função $ b(x) $ de \textit{Kindlmann e Durkin}~\cite{gordon}, o que faz dele um método mais automático. Definir $ b(x) $ exige que o usuário defina o comportamento da opacidade do volume na fronteira. Como o objetivo desta dissertação é realçar todas as fronteiras, definiu-se este comportamento por uma gaussiana, permitindo ao usuário ajustar sua amplitude e largura. Dessa forma, quando a largura da fronteira é alterada, as amplitudes máximas da função de transferência não mudam, diferente do que ocorre quando se utiliza $ b(x) $.
	
	Da mesma forma que \textit{Kindlmann e Durkin}~\cite{gordon} utilizam o mesmo $ \sigma $ para todo o volume, a largura da gaussiana definida pelo usuário nesta dissertação é a mesma para todas as fronteiras. No entanto, como argumentado no Capítulo~\ref{ch:my}, cada fronteira deveria possuir seu próprio $ \sigma $. Então, como trabalho futuro, deseja-se incorporar esse conceito ao método proposto, explorando os arcos de $ t(v) $ de forma que a espessura de cada fronteira varie de acordo com a largura do arco ao qual pertence. Além disso, é desejável que o formato do arco também influencie a amplitude da gaussiana.
	
	Ainda como trabalho futuro, deseja-se criar restrições de opacidade nas fronteiras, priorizando as fronteiras internas ao volume, de forma que não haja oclusão de uma fronteira por outra. Da mesma forma, espera-se encontrar um meio de identificar apenas uma fronteira ao invés de múltiplas fronteiras paralelas, quando estas não se comportam de forma ideal.