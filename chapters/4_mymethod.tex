% -*- coding: utf-8; -*-

\chapter{Geração Baseada em Derivadas Médias}
\label{my}
	Este capítulo descreve o método proposto nessa dissertação para gerar funções de transferência automáticas, baseado no trabalho de \textit{Kindlmann e Durkin}~\cite{gordon}, explicado no capítulo anterior.

\section{Detecção de Fronteiras}
\label{my.deriv}
	No capítulo~\ref{gordon} foi mostrado que a partir da primeira derivada máxima e segunda derivada igual a zero, encontra-se a posição exata do centro da fronteira. Através das derivadas médias amostradas por todo o volume, descobre-se a distância média dos voxels de valor $ v $ à sua fronteira mais próxima. e que mesmo quando 
	
	O modelo de fronteiras proposto possui $ f''(x) = 0 $ em 3 momentos. Por isso utilizar $ f''(x) = 0 $ para identificar fronteiras pode ser dúbio. Claro que como $ f''(x) $ é derivada de $ f'(x) $ e FT é definida por máximo em $ f'(x) $ e zero em $ f''(x) $, sabemos que isso nunca ocorrerá nos pontos errados de $ f''(x) = 0 $. No entanto essa garantia é perdida no momento que utilizamos valores médios, isto é $ g(v) $ e $ h(v) $ no lugar de $ f'(x) $ e $ f''(x) $.
	
	Ao avaliar a terceira derivada de $ f(x) $ no entanto, percebemos uma característica interessante. O valor mínimo se encontra exatamente no centro da fronteira.
    
\subsection{Malhas estruturadas}
\label{my.struct}
	Explicar malha estruturada...

\subsection{Malhas não estruturadas}
\label{my.nonstruct}
	Explicar malha não estruturada...

\section{Geração da função de transferência}
\label{my.tf}
	Texto...